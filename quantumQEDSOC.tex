%% LyX 1.6.6.1 created this file.  For more info, see http://www.lyx.org/.
%% Do not edit unless you really know what you are doing.
\documentclass[english]{paper}
\usepackage[T1]{fontenc}
\usepackage[latin9]{inputenc}
\usepackage[a4paper]{geometry}
\geometry{verbose,tmargin=0.5in,bmargin=1.2in,lmargin=0.5in,rmargin=0.5in,headsep=0.2in}
\usepackage{amstext}
\usepackage{graphicx}
\usepackage{amssymb}
\usepackage{esint}

\makeatletter
%%%%%%%%%%%%%%%%%%%%%%%%%%%%%% User specified LaTeX commands.
\def\pa{\partial}
\def\r{\textbf{r}}
\def\k{\textbf{k}}
\def\q{\textbf{q}}
\def\p{\textbf{p}}
\def\R{\textbf{R}}
\def\s{\sigma}
\def\u{\uparrow}
\def\d{\downarrow}

\makeatother

\usepackage{babel}

\begin{document}

\title{Quantum version of the model, by Lin Dong, 10/31/13}

\maketitle
In this note, we consider the quantum version of the cavity-assisted
SOC model, where photon and atom field is not treated as a mean field
variable. 

Using bare atomic psedo-spin operater $\Psi_{\sigma}({\bf r})$ and
photon field operater $c$ (without explicit time dependence), we
can write the atom-cavity Hamiltonian as, \begin{eqnarray}
H & = & \sum_{\sigma}\int d{\bf r}\left[\Psi_{\sigma}^{\dagger}({\bf r})\left(\frac{\hbar^{2}\hat{{\bf k}}^{2}}{2m}+\epsilon_{\sigma}^{0}\right)\Psi_{\sigma}({\bf r})\right]\nonumber \\
 & + & \frac{\Omega}{2}\int d{\bf r}e^{+2i\hbar q_{r}z}\Psi_{\uparrow}^{\dagger}({\bf r})\Psi_{\downarrow}({\bf r})\tilde{c}e^{+i\omega_{R}t}\nonumber \\
 & + & \frac{\Omega}{2}\int d{\bf r}e^{-2i\hbar q_{r}z}\tilde{c}^{\dagger}\Psi_{\downarrow}^{\dagger}({\bf r})\Psi_{\uparrow}({\bf r})e^{-i\omega_{R}t}\nonumber \\
 & + & i\varepsilon_{p}(\tilde{c}^{\dagger}e^{-i\omega_{p}t}-\tilde{c}e^{+i\omega_{p}t})+\hbar\omega_{c}\tilde{c}^{\dagger}\tilde{c}.\end{eqnarray}
We work in rotating frame $c=\tilde{c}e^{+i\omega_{p}t}$ and gauge
transformation $\tilde{\psi}_{\uparrow}=\Psi_{\uparrow}e^{-i\hbar q_{r}z}$,
$\tilde{\psi}_{\downarrow}=\Psi_{\downarrow}e^{+i\hbar q_{r}z}$ .
After unitary transformation of origional Hamiltonian, we can write
\begin{eqnarray}
\mathcal{H}_{\text{eff}} & = & \int d{\bf r}\left(\begin{array}{cc}
\tilde{\psi}_{\uparrow}^{\dagger}({\bf r}) & \tilde{\psi}_{\downarrow}^{\dagger}({\bf r})\end{array}\right)\left[\frac{\hbar^{2}\hat{{\bf k}}^{2}}{2m}+\frac{\hbar^{2}}{m}q_{r}\hat{k_{z}}\sigma_{z}+\tilde{\delta}\sigma_{z}\right]\left(\begin{array}{c}
\tilde{\psi}_{\uparrow}({\bf r})\\
\tilde{\psi}_{\downarrow}({\bf r})\end{array}\right)\nonumber \\
 & + & \int d{\bf r}\frac{\Omega}{2}\tilde{\psi}_{\uparrow}^{\dagger}({\bf r})\tilde{\psi}_{\downarrow}({\bf r})\tilde{c}+\int d{\bf r}\frac{\Omega}{2}c^{\dagger}\tilde{\psi}_{\downarrow}^{\dagger}({\bf r})\tilde{\psi}_{\uparrow}({\bf r})\nonumber \\
 & + & i\varepsilon_{p}(c^{\dagger}-c)-\hbar\delta_{c}c^{\dagger}c\end{eqnarray}
where $\delta_{c}=\omega_{p}-\omega_{c}$ and we have neglected constant
energy $\frac{\hbar^{2}q_{r}^{2}}{2m}$ and incorprated energy shift
of two-photon detunning into $\tilde{\delta}=\delta_{R}/2+(\epsilon_{\uparrow}^{0}-\epsilon_{\downarrow}^{0})$
and $\delta_{R}=\omega_{p}-\omega_{R}$. 

As we consider external pumping, we shall deal with disspation of
cavity decay in terms of density operator approach. Generally dissipation
process is modeled by Liouvillean term $\mathcal{L}$ appearing in
the master equation (where we only consider cavity decay and neglect
effect of spontaneous emission), \begin{equation}
\dot{\rho}=\frac{1}{i\hbar}[\mathcal{H}_{\text{eff}},\rho]+\mathcal{L}\rho\end{equation}
where \begin{equation}
\mathcal{L}\rho=\kappa(2c\rho c^{\dagger}-c^{\dagger}c\rho-\rho c^{\dagger}c).\end{equation}


For the homogeneous system, momentum operator $\hat{{\bf k}}$ commutes
with $\mathcal{H}_{\text{eff}}$, and we can decouple each momentum
eigenstate using the ansartz $\tilde{\psi}_{\sigma}({\bf k})=e^{i{\bf k}\cdot{\bf r}}\psi_{\sigma}$.
For given momentum (as a good quantum number), we work with the Hamiltonian
in subspace of ${\bf k}$, \begin{eqnarray}
\mathcal{H}_{\text{eff}}({\bf k}) & = & \sum_{\sigma}\left(\frac{{\bf k}^{2}}{2m}+\alpha\frac{q_{r}k_{z}}{m}+\alpha\tilde{\delta}\right)\psi_{\sigma}^{\dagger}\psi_{\sigma}+\frac{\Omega}{2}\left(\psi_{\uparrow}^{\dagger}\psi_{\downarrow}c+c^{\dagger}\psi_{\downarrow}^{\dagger}\psi_{\uparrow}\right)\nonumber \\
 & + & i\varepsilon_{p}(c^{\dagger}-c)-\delta_{c}c^{\dagger}c\end{eqnarray}
where $\alpha=\pm1$ for $\sigma=\uparrow,\downarrow$ respectively. 

Then, we iron out the commutator explicitly as, \begin{eqnarray}
[\mathcal{H}_{\text{eff}}({\bf k}),\rho] & = & \left(\frac{{\bf k}^{2}}{2m}+\frac{q_{r}k_{z}}{m}+\tilde{\delta}\right)\left(\psi_{\uparrow}^{\dagger}\psi_{\uparrow}\rho-\rho\psi_{\uparrow}^{\dagger}\psi_{\uparrow}\right)+\left(\frac{{\bf k}^{2}}{2m}-\frac{q_{r}k_{z}}{m}-\tilde{\delta}\right)\left(\psi_{\downarrow}^{\dagger}\psi_{\downarrow}\rho-\rho\psi_{\downarrow}^{\dagger}\psi_{\downarrow}\right)\\
 & + & \frac{\Omega}{2}\left(\psi_{\uparrow}^{\dagger}\psi_{\downarrow}c\rho+c^{\dagger}\psi_{\downarrow}^{\dagger}\psi_{\uparrow}\rho-\rho\psi_{\uparrow}^{\dagger}\psi_{\downarrow}c-\rho c^{\dagger}\psi_{\downarrow}^{\dagger}\psi_{\uparrow}\right)+i\varepsilon_{p}\left(c^{\dagger}\rho-c\rho-\rho c^{\dagger}+\rho c\right)-\delta_{c}\left(c^{\dagger}c\rho-\rho c^{\dagger}c\right).\end{eqnarray}


For few photons, the quantum phenomena shall be more prominant.



To this end, we choose our basis states as $|n;\sigma\rangle$ where
$n=0,1,2,...N$ where $N$ is the truncation number and $\sigma=\uparrow,\downarrow$.
Our goal is to calculate matrix elements of density operator under
this basis $\langle m;\sigma|\rho|n;\sigma'\rangle\equiv\rho_{mn}^{\sigma\sigma'}$.
Rules for creation and annilation operators are \begin{eqnarray*}
c|n;\sigma'\rangle & = & \sqrt{n}|n-1;\sigma'\rangle,\quad c^{\dagger}|n;\sigma'\rangle=\sqrt{n+1}|n+1;\sigma'\rangle\\
\langle m;\sigma|c & = & \sqrt{m+1}\langle m+1;\sigma|,\langle m;\sigma|c^{\dagger}=\sqrt{m}\langle m-1;\sigma|.\end{eqnarray*}
In the $\mathcal{H}_{\text{eff}}$, Raman term involves spin flipping
and cavity photon coupling, whereas the rest of terms involve cavity
photon coupling only. For arbitrary state, we have\begin{eqnarray}
\frac{d}{dt}\rho_{mn}^{\sigma\sigma'} & = & \frac{h_{1}}{i}\left(\delta_{\sigma\uparrow}-\delta_{\sigma'\uparrow}\right)\rho_{mn}^{\sigma\sigma'}+\frac{h_{2}}{i}\left(\delta_{\sigma\downarrow}-\delta_{\sigma'\downarrow}\right)\rho_{mn}^{\sigma\sigma'}\\
 & + & \frac{\Omega}{2i}\left(\delta_{\sigma\uparrow}\sqrt{m+1}\rho_{m+1n}^{\bar{\sigma}\sigma'}+\delta_{\sigma\downarrow}\sqrt{m}\rho_{m-1n}^{\bar{\sigma}\sigma'}-\delta_{\sigma'\uparrow}\sqrt{n+1}\rho_{mn+1}^{\sigma\bar{\sigma'}}-\delta_{\sigma'\downarrow}\sqrt{n}\rho_{mn-1}^{\sigma\bar{\sigma'}}\right)\\
 & + & \varepsilon_{p}\left(\sqrt{m}\rho_{m-1n}^{\sigma\sigma'}-\sqrt{m+1}\rho_{m+1n}^{\sigma\sigma'}+\sqrt{n}\rho_{mn-1}^{\sigma\sigma'}-\sqrt{n+1}\rho_{mn+1}^{\sigma\sigma'}\right)\\
 & - & \frac{\delta_{c}}{i}\left(m-n\right)\rho_{mn}^{\sigma\sigma'}\\
 & + & \kappa\left(2\sqrt{m+1}\sqrt{n+1}\rho_{m+1n+1}^{\sigma\sigma'}-(m+n)\rho_{mn}^{\sigma\sigma'}\right)\end{eqnarray}
where $h_{1}=\frac{k^{2}}{2}+q_{r}k+\tilde{\delta}$, $h_{2}=\frac{k^{2}}{2}-q_{r}k-\tilde{\delta}$,
and $\bar{\sigma}$ represents the flip-spin value, for instance $\bar{\uparrow}=\downarrow$,
$\bar{\downarrow}=\uparrow$ . 



Explicitly, and without loss of generality, the matrix elements are,\begin{eqnarray}
\frac{d}{dt}\langle m;\uparrow|\rho|n;\uparrow\rangle & = & \frac{1}{i\hbar}\frac{\Omega}{2}\left(\sqrt{m+1}\langle m+1;\downarrow|\rho|n;\uparrow\rangle-\sqrt{n+1}\langle m;\uparrow|\rho|n+1;\downarrow\rangle\right)\nonumber \\
 & + & \frac{1}{i\hbar}i\varepsilon_{p}\left(\sqrt{m}\langle m-1;\uparrow|\rho|n;\uparrow\rangle-\sqrt{m+1}\langle m+1;\uparrow|\rho|n;\uparrow\rangle-\sqrt{n+1}\langle m;\uparrow|\rho|n+1;\uparrow\rangle+\sqrt{n}\langle m;\uparrow|\rho|n-1;\uparrow\rangle\right)\nonumber \\
 & - & \frac{1}{i\hbar}\delta_{c}\left(m-n\right)\langle m;\uparrow|\rho|n;\uparrow\rangle\nonumber \\
 & + & \kappa\left(2\sqrt{m+1}\sqrt{n+1}\langle m+1;\uparrow|\rho|n+1;\uparrow\rangle-(m+n)\langle m;\uparrow|\rho|n;\uparrow\rangle\right)\\
\frac{d}{dt}\langle m;\downarrow|\rho|n;\downarrow\rangle & = & \frac{1}{i\hbar}\frac{\Omega}{2}\left(\sqrt{m}\langle m-1;\uparrow|\rho|n;\downarrow\rangle-\sqrt{n}\langle m;\downarrow|\rho|n-1;\uparrow\rangle\right)\nonumber \\
 & + & \frac{1}{i\hbar}i\varepsilon_{p}\left(\sqrt{m}\langle m-1;\downarrow|\rho|n;\downarrow\rangle-\sqrt{m+1}\langle m+1;\downarrow|\rho|n;\downarrow\rangle-\sqrt{n+1}\langle m;\downarrow|\rho|n+1;\downarrow\rangle+\sqrt{n}\langle m;\downarrow|\rho|n-1;\downarrow\rangle\right)\nonumber \\
 & - & \frac{1}{i\hbar}\delta_{c}\left(m-n\right)\langle m;\downarrow|\rho|n;\downarrow\rangle\nonumber \\
 & + & \kappa\left(2\sqrt{m+1}\sqrt{n+1}\langle m+1;\downarrow|\rho|n+1;\downarrow\rangle-(m+n)\langle m;\downarrow|\rho|n;\downarrow\rangle\right)\\
\frac{d}{dt}\langle m;\uparrow|\rho|n;\downarrow\rangle & = & +\frac{1}{i\hbar}2\left(q_{r}k_{z}+\tilde{\delta}\right)\langle m;\uparrow|\rho|n;\downarrow\rangle\nonumber \\
 & + & \frac{1}{i\hbar}\frac{\Omega}{2}\left(\sqrt{m+1}\langle m+1;\downarrow|\rho|n;\downarrow\rangle-\sqrt{n}\langle m;\uparrow|\rho|n-1;\uparrow\rangle\right)\nonumber \\
 & + & \frac{1}{i\hbar}i\varepsilon_{p}\left(\sqrt{m}\langle m-1;\uparrow|\rho|n;\downarrow\rangle-\sqrt{m+1}\langle m+1;\uparrow|\rho|n;\downarrow\rangle-\sqrt{n+1}\langle m;\uparrow|\rho|n+1;\downarrow\rangle+\sqrt{n}\langle m;\uparrow|\rho|n-1;\downarrow\rangle\right)\nonumber \\
 & - & \frac{1}{i\hbar}\delta_{c}\left(m-n\right)\langle m;\uparrow|\rho|n;\downarrow\rangle\nonumber \\
 & + & \kappa\left(2\sqrt{m+1}\sqrt{n+1}\langle m+1;\uparrow|\rho|n+1;\downarrow\rangle-(m+n)\langle m;\uparrow|\rho|n;\downarrow\rangle\right)\\
\frac{d}{dt}\langle m;\downarrow|\rho|n;\uparrow\rangle & = & -\frac{1}{i\hbar}2\left(q_{r}k_{z}+\tilde{\delta}\right)\langle m;\downarrow|\rho|n;\uparrow\rangle\nonumber \\
 & + & \frac{1}{i\hbar}\frac{\Omega}{2}\left(\sqrt{m}\langle m-1;\uparrow|\rho|n;\uparrow\rangle-\sqrt{n+1}\langle m;\downarrow|\rho|n+1;\downarrow\rangle\right)\nonumber \\
 & + & \frac{1}{i\hbar}i\varepsilon_{p}\left(\sqrt{m}\langle m-1;\downarrow|\rho|n;\uparrow\rangle-\sqrt{m+1}\langle m+1;\downarrow|\rho|n;\uparrow\rangle-\sqrt{n+1}\langle m;\downarrow|\rho|n+1;\uparrow\rangle+\sqrt{n}\langle m;\downarrow|\rho|n-1;\uparrow\rangle\right)\nonumber \\
 & - & \frac{1}{i\hbar}\delta_{c}(m-n)\langle m;\downarrow|\rho|n;\uparrow\rangle\nonumber \\
 & + & \kappa\left(2\sqrt{m+1}\sqrt{n+1}\langle m+1;\downarrow|\rho|n+1;\uparrow\rangle-(m+n)\langle m;\downarrow|\rho|n;\uparrow\rangle\right)\end{eqnarray}
Notationally, \begin{equation}
[\rho_{mn}^{\sigma\sigma'}]_{(2N+2)\times(2N+2)}=\left(\begin{array}{cc}
[\rho_{mn}^{\uparrow\uparrow}]_{(N+1)\times(N+1)} & [\rho_{mn}^{\uparrow\downarrow}]_{(N+1)\times(N+1)}\\
{}[\rho_{mn}^{\downarrow\uparrow}]_{(N+1)\times(N+1)} & [\rho_{mn}^{\downarrow\downarrow}]_{(N+1)\times(N+1)}\end{array}\right)\end{equation}
In the column of $[\rho_{mn}^{\sigma\sigma'}]_{(N+1)^{2}\times1}$,
the $k$th element is accessed as $k=m(N+1)+(n+1)$. Since we have
finite truncation number $N$, we shall ignore terms involving $|N+1;\sigma\rangle$
or $|-1;\sigma\rangle$ generated by photon creation or annilation
operators, as they either hit the ceiling or floor of our truncated
Hilbert space. We then further write EOM of density matrix as\begin{equation}
\frac{d}{dt}\left(\begin{array}{c}
[\rho_{mn}^{\uparrow\uparrow}]_{(N+1)^{2}\times1}\\
{}[\rho_{mn}^{\uparrow\downarrow}]_{(N+1)^{2}\times1}\\
{}[\rho_{mn}^{\downarrow\uparrow}]_{(N+1)^{2}\times1}\\
{}[\rho_{mn}^{\downarrow\downarrow}]_{(N+1)^{2}\times1}\end{array}\right)_{(2N+2)^{2}\times1}=\left(\begin{array}{cccc}
[M_{mn}^{\uparrow\uparrow}] & [S_{mn}^{1}] & [S_{mn}^{2}] & 0\\
{}[S_{mn}^{3}] & [M_{mn}^{\uparrow\downarrow}] & 0 & [S_{mn}^{4}]\\
{}[S_{mn}^{5}] & 0 & [M_{mn}^{\downarrow\uparrow}] & [S_{mn}^{6}]\\
0 & [S_{mn}^{7}] & [S_{mn}^{8}] & [M_{mn}^{\downarrow\downarrow}]\end{array}\right)_{(2N+2)^{2}\times(2N+2)^{2}}\left(\begin{array}{c}
[\rho_{mn}^{\uparrow\uparrow}]\\
{}[\rho_{mn}^{\uparrow\downarrow}]\\
{}[\rho_{mn}^{\downarrow\uparrow}]\\
{}[\rho_{mn}^{\downarrow\downarrow}]\end{array}\right)_{(2N+2)^{2}\times1}\label{eq:EOM_rho_matrix}\end{equation}
 

For $N=1$ and given initial state $\rho_{00}^{\uparrow\uparrow}(t=0)=1$
and else $0$, after some transient time, all matrix elements (16
of them) reach steady state, as typically shown in the figure on the
left, (absolute values have been taken)

\includegraphics[scale=0.5]{typical_rho}\includegraphics[scale=0.5]{trace_rho}

On the right, we plot the trace of density matrix $\text{Tr}[\rho_{mn}^{\sigma\sigma'}]_{(2N+2)\times(2N+2)}=\text{Tr}[\rho_{mn}^{\uparrow\uparrow}]_{(N+1)\times(N+1)}+\text{Tr}[\rho_{mn}^{\downarrow\downarrow}]_{(N+1)\times(N+1)}$
as a function of time. As expected, the trace remains strictly to
be 1 throughout time evolution. However, the trace of density matrix
square $\text{Tr}[\rho^{2}]$ does not monotonically decreases with
time, 

\includegraphics[scale=0.5]{trace_rho_square}\includegraphics[scale=0.5]{trace_rho_square_kappa0}

This means initially the system is a pure state $|0;\uparrow\rangle$
where $\text{Tr}\rho^{2}=1$, however due to cavity decay, the system
becomes a mixed state $\text{Tr}\rho^{2}<1$; for comparison, we can
set $\kappa=0$ and evolve the system, to get the plot on the right,
where we have pure state all the time. 

Steady state solution can be obtained by directly requiring the RHS
of Eq. \ref{eq:EOM_rho_matrix} to be zero. Note that, one of the
equations is redundant, due to the constraint of $\text{Tr}[\rho]=1$.
The expectation value of any observable can be obtained by tracing
the product of density operator and observable operator, namely $\langle\hat{O}\rangle=\text{Tr}[\rho\hat{O}]$. 
\begin{itemize}
\item One simple case is the expectation value of photon number $\langle n\rangle=\text{Tr}[\rho n]=\text{Tr}\left[[\rho_{mn}^{\sigma\sigma'}]_{(2N+2)\times(2N+2)}\times[\hat{n}_{mn}^{\sigma\sigma'}]_{(2N+2)\times(2N+2)}\right]$
where $\langle m;\sigma|c^{\dagger}c|n;\sigma'\rangle=n\delta_{mn}\delta_{\sigma\sigma'}$.
And, photon number fluctuation $\frac{\langle(\Delta n)^{2}\rangle}{\langle n\rangle}=\frac{\langle n^{2}\rangle-\langle n\rangle^{2}}{\langle n\rangle}$,
where $\langle m;\sigma|c^{\dagger}cc^{\dagger}c|n;\sigma'\rangle=n^{2}\delta_{mn}\delta_{\sigma\sigma'}$.
If $\frac{\langle(\Delta n)^{2}\rangle}{\langle n\rangle}>1$, it
is super-Poissonian; if $\frac{\langle(\Delta n)^{2}\rangle}{\langle n\rangle}=1$,
it is Poissonian; if $\frac{\langle(\Delta n)^{2}\rangle}{\langle n\rangle}<1$,
it is sub-Poissonian. 
\end{itemize}




In the following plots, we choose parameters as $\tilde{\delta}=0$,
$k_{z}=k_{x}=k_{y}=0$, $\kappa=1=\varepsilon_{p}=\Omega$, $q_{r}=0.22$,
$\delta_{c}=1$, and $N=6$. 

\includegraphics[scale=0.35]{photon_decay}\includegraphics[scale=0.35]{photon_pump}\includegraphics[scale=0.35]{photon_deltac}

\includegraphics[scale=0.35]{/Users/donglin/Documents/Research/QED_SOC_Quantum/photonFluc_decay}\includegraphics[scale=0.35]{/Users/donglin/Documents/Research/QED_SOC_Quantum/photonFluc_pump}\includegraphics[scale=0.35]{/Users/donglin/Documents/Research/QED_SOC_Quantum/photonFluc_deltac}

\includegraphics[scale=0.35]{fluc_kappa}\includegraphics[scale=0.35]{fluc_epsilonp}\includegraphics[scale=0.35]{fluc_deltac}

\includegraphics[scale=0.35]{photonN_Omega}\includegraphics[scale=0.35]{photonN_Omega_Delta_1n}\includegraphics[scale=0.35]{photonN_Omega_Delta_n}




\begin{itemize}
\item Reduced density matrix for photon, $\rho_{\text{photon}}=\text{Tr}_{\text{atom}}[\rho]=[\rho_{mn}^{\uparrow\uparrow}]_{(2N+2)\times(2N+2)}+[\rho_{mn}^{\downarrow\downarrow}]_{(2N+2)\times(2N+2)}$;
Reduced density matrix for atom, $\rho_{\text{atom}}=\text{Tr}_{\text{photon}}[\rho]=\left(\begin{array}{cc}
\sum_{n}\rho_{nn}^{\uparrow\uparrow} & \sum_{n}\rho_{nn}^{\uparrow\downarrow}\\
\sum_{n}\rho_{nn}^{\downarrow\uparrow} & \sum_{n}\rho_{nn}^{\downarrow\downarrow}\end{array}\right)$.
\end{itemize}
\includegraphics[scale=0.35]{trace_rho_photon_square_Omega_kappa}\includegraphics[scale=0.35]{trace_rho_square_Omega_epsilonp}\includegraphics[scale=0.35]{trace_rho_square_Omega_deltac}
\begin{itemize}
\item $p(n)$ is the probability of finding $n$ photon. $p(n)=\langle n|\hat{\rho}_{\text{photon}}|n\rangle=\rho_{nn}^{\uparrow\uparrow}+\rho_{nn}^{\downarrow\downarrow}$.
For coherent light, $p_{\text{coh}}(n)=\frac{\langle n\rangle^{n}}{n!}e^{-\langle n\rangle}$;
and for thermal light, $p_{\text{th}}(n)=\frac{1}{1+\langle n\rangle}(\frac{\langle n\rangle}{1+\langle n\rangle})^{n}$.
In the following plots, we choose parameters as $\tilde{\delta}=0$,
$k_{z}=k_{x}=k_{y}=0$, $\kappa=1=\varepsilon_{p}=\Omega$, $q_{r}=0.22$,
$\delta_{c}=1$, and $N=10$. 
\end{itemize}


\includegraphics[scale=0.35]{pn_reducedrho_epsilonp_1}\includegraphics[scale=0.35]{pn_reducedrho_epsilonp_2}\includegraphics[scale=0.35]{pn_reducedrho_epsilonp_3}
\begin{itemize}
\item The expectation value of $\mathcal{H}_{\text{eff}}({\bf k})$ is 
\end{itemize}
\begin{eqnarray}
\langle m;\sigma|\mathcal{H}_{\text{eff}}({\bf k})|n;\sigma'\rangle & = & h_{1}\delta_{mn}\delta_{\sigma\sigma'}\delta_{\sigma'\uparrow}+h_{2}\delta_{mn}\delta_{\sigma\sigma'}\delta_{\sigma'\downarrow}\\
 & + & \frac{\Omega}{2}\left(\sqrt{n}\delta_{mn-1}\delta_{\sigma\bar{\sigma'}}\delta_{\sigma'\downarrow}+\sqrt{n+1}\delta_{mn+1}\delta_{\sigma\bar{\sigma'}}\delta_{\sigma'\uparrow}\right)\\
 & + & i\varepsilon_{p}\left(\sqrt{n+1}\delta_{mn+1}\delta_{\sigma\sigma'}-\sqrt{n}\delta_{mn-1}\delta_{\sigma\sigma'}\right)-\delta_{c}n\delta_{mn}\delta_{\sigma\sigma'}\end{eqnarray}
where $h_{1}=\frac{k^{2}}{2}+q_{r}k+\tilde{\delta}$, $h_{2}=\frac{k^{2}}{2}-q_{r}k-\tilde{\delta}$,
and $\bar{\sigma}$ represents the flip-spin value, for instance $\bar{\uparrow}=\downarrow$,
$\bar{\downarrow}=\uparrow$ . We denote $H_{C}=i\varepsilon_{p}\left(\sqrt{n+1}\delta_{mn+1}\delta_{\sigma\sigma'}-\sqrt{n}\delta_{mn-1}\delta_{\sigma\sigma'}\right)-\delta_{c}n\delta_{mn}\delta_{\sigma\sigma'}$.
If we include $H_{C}$ in $\langle H_{\text{eff}}\rangle$, we have

\includegraphics[scale=0.35]{energyOmega1}\includegraphics[scale=0.35]{energyOmega5}\includegraphics[scale=0.35]{energyOmega6}

If we don't include $H_{C}$ in $\langle H_{\text{eff}}\rangle$,
we have

\includegraphics[scale=0.35]{energyOmega1_bare}\includegraphics[scale=0.35]{energyOmega5_bare}\includegraphics[scale=0.35]{energyOmega6_bare}
\begin{itemize}
\item Negativity for mixed state is defined as $\mathcal{N}(\rho)=\frac{||\rho^{T_{atom}}||_{1}-1}{2}=\frac{\left(\sum\text{eig}[\rho^{T_{\text{atom}}}]\right)-1}{2}=\sum_{i}\frac{|\lambda_{i}|-\lambda_{i}}{2}$
where $\lambda_{i}$ are all the eigenvalues of $\rho^{T_{atom}}$.
Namely, nagativity is the absolute sum of negative eigenvalues of
$\rho^{T_{atom}}$, which vanishes for unentangled states. Also, $\rho^{T_{atom}}$
stands for partial transpose of density matrix with respect to atom
party, $\rho^{T_{atom}}=\left(\begin{array}{cc}
[\rho_{mn}^{\uparrow\uparrow}]^{T} & [\rho_{mn}^{\uparrow\downarrow}]^{T}\\
{}[\rho_{mn}^{\downarrow\uparrow}]^{T} & [\rho_{mn}^{\downarrow\downarrow}]^{T}\end{array}\right)$. 
\end{itemize}
\includegraphics[scale=0.35]{negativity_kappa}\includegraphics[scale=0.35]{negativity_epsilonp}\includegraphics[scale=0.35]{negativity_deltac}
\end{document}
